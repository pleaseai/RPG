\vspace*{-5pt}
\section{Main Results}
\label{sec:result}
\vspace*{-3pt}
\begin{table*}[t]
\centering
\caption{Performance of agent frameworks and model backbones on RepoCraft. "Nov." denotes the novelty rate; the number in parentheses is Novel/Total, where Novel is the number of novel functionalities and Total is the total number of planned functionalities. Gold Projects are used as a confidence ablation for the automatic evaluation pipeline, and per-repository detailed results are reported in Appendix~\ref{app:detailed_result}.}
\label{tab:main_results}
\resizebox{\textwidth}{!}{%
\begin{tabular}{llcccccc}
\toprule
\textbf{Agent} & \textbf{Model} & \textbf{Cov.} (\%) $\uparrow$ & \textbf{Nov.} (\%) (Novel/Total) $\uparrow$ & \textbf{Pass. / Vot.} (\%) $\uparrow$ & \textbf{Files} $\uparrow$ & \textbf{LOC} $\uparrow$ & \textbf{Tokens} $\uparrow$ \\
\midrule
\multirow{2}{*}{MetaGPT}
  & o3-mini     & 16.6 & 0.0 (0.0/24.8) & 4.5 / 10.2 & 2.3 & 225.3 & 2180.3 \\
  & Qwen3-Coder & 17.1 & 0.0 (0.0/32.7) & 3.2 / 9.4 & 8.5 & 326.5 & 3369.3 \\
\midrule
\multirow{2}{*}{ChatDev}
  & o3-mini     & 18.3 & 9.2 (3.0/32.8) & 2.6 / 10.5 & 5.8 & 410.3 & 4458 \\
  & Qwen3-Coder & 22.1 & 3.9 (1.5/38.3) & 6.9 / 11.6 & 6.3 & 540.7 & 5422.2 \\
\midrule
\multirow{2}{*}{OpenHands}
  & o3-mini     & 22.0 & 0.3 (0.1/36.5) & 5.1 / 16.9 & 9.8 & 292.2 & 2712.8 \\
  & Qwen3-Coder & 21.7 & 0.0 (0.0/33.7) & 5.8 / 11.2 & 8.3 & 458.0 & 4778.3 \\
\midrule
\multirow{2}{*}{Paper2Code}
  & o3-mini     & 21.7 & 5.2 (2.1/40.0) & 6.0 / 15.8 & 7.2 & 547.7 & 5920.8 \\
  & Qwen3-Coder & 30.2 & 5.5 (4.0/73.8)& 4.9 / 15.9 & 8.8 & 1365.2 & 14,555.0 \\
\midrule
Codex CLI        & o3 pro             & 28.4 & 0.0 (0.0/48.5) & 11.0 / 20.0 & 5.3 & 611.5 & 6248.5 \\
Gemini CLI       & gemini 2.5 pro     & 42.0 & 0.6 (0.8/132.7) & 14.5 / 37.9 & 15.2 & 1484.8 & 14,922.2 \\
Claude Code CLI  & claude 4 sonnet    & 54.2 & 6.7 (41.6/621.0) & 33.9 / 52.5 & 33.3 & 10,586.7 &  105,236.2 \\
\midrule
\rowcolor{gray!20}
\multirow{1}{*}{Gold Projects} 
  & Human Developers 
  & - & - 
  & 81.0 / 92.0 
  & 345 
  & 97,819.7 
  & 951,614 \\
\midrule
\multirow{2}{*}{\textbf{\ours}}
  & o3-mini     & \textbf{81.5} & \textbf{13.6} (\textbf{151.5}/\textbf{1114.2}) & \textbf{69.7 / 75.0} & \textbf{271.5} & \textbf{23,977.3} & \textbf{260,761.2} \\
  & Qwen3-Coder & \textbf{75.1} & \textbf{9.2} (\textbf{108.3}/\textbf{1173.3}) & \textbf{57.3 / 68.0} & \textbf{389.0} & \textbf{36,941.0} & \textbf{445,511.8} \\
\bottomrule
\end{tabular}%
}
\end{table*}
\vspace*{-3pt}
\begin{figure}[htbp]
\centering
\vspace{0pt}
\includegraphics[width=\linewidth]{figs/dependency.pdf}
\captionof{figure}{Illustration of dependencies in the repository generated by Qwen3-Coder on \texttt{MLKit-Py}, showing (1) the repository skeleton at the folder/file level, (2) inter-module data flows, and (3) class and function dependencies.}
\label{fig:dependency}
\end{figure}

\paragraph{\graph{} enables richer functionality and larger repositories.}
\ours{} demonstrates that \graph{}–guided planning yields repositories of substantially greater scale, diversity, and novelty than existing approaches. On RepoCraft, it achieves up to 81.5\% coverage with \texttt{o3-mini}, representing a 27.3\% absolute improvement over the strongest baseline (Claude Code at 54.2\%). Beyond covering the required functionality, \ours{} also exhibits strong innovation, attaining novelty rates of 11–13\% with over 100 new functionalities, whereas most baselines contribute fewer than 10. In terms of repository size, \ours{} with \texttt{Qwen3-Coder} generates 36K LOC and 445K tokens, corresponding to 3.9× the code scale of Claude Code and about 68× that of other baselines. Among these approaches, \ours{} is the closest to human-developed Gold Projects, underscoring that \graph{} serves as the key structured representation for building repositories that are larger, more diverse, and closer to real-world software complexity.
\vspace*{-2pt}
\paragraph{\graph{} enhances reasoning consistency and structural fidelity.}
Beyond scale, \ours{} delivers substantially higher correctness and stability. To ensure reliability, we first validate the automatic localization and validation pipeline on human-developed Gold Projects, where it achieves 81.0\% pass rate and 92.0\% voting agreement, establishing the ceiling under our test harness. Under the same protocol, \ours{} attains a 69.7\% pass rate with \texttt{o3-mini}, an absolute improvement of 35.8\% compared to the strongest baseline (Claude Code at 33.9\%). These results indicate that \graph{} serves as a structured reasoning representation that enforces modular boundaries and functional contracts, thereby supporting coherent planning and yielding repositories that more faithfully realize intended specifications.
\vspace*{-2pt}
\paragraph{\graph{} induces complex data flows and dependencies.}
To illustrate the capacity of \graph{}–guided planning for generating complex repositories, we visualize \ours{} with \texttt{Qwen3-Coder} on the \texttt{MLKit-Py} task. At the file level, \graph{} organizes a coherent folder hierarchy; at the module level, inter-module flows define execution pipelines from \texttt{data\_lifecycle} through \texttt{clustering} and \texttt{models} to \texttt{evaluation}; and at the function level, inheritance and invocation edges capture intricate class interactions. These results show that \graph{} induces layered dependencies and coordinated execution, enabling repositories with both structural complexity and internal coherence.