% \vspace*{-5pt}
\section{Analysis}
\label{sec:analysis}
\subsection{Representational Efficiency}

\begin{wraptable}{r}{0.48\textwidth}
\vspace{-6pt}
\centering
\caption{Efficiency for repository understanding on SWE-bench Verified. Steps and Cost are averaged over tasks. Eff. is defined as $\mathrm{Acc@5}/\mathrm{Cost}$. Additional results are provided in Appendix~\ref{app:cost_analysis}.}
\label{tab:efficiency_analysis}
\footnotesize
\setlength{\tabcolsep}{4pt}

\resizebox{\linewidth}{!}{%
\begin{tabular}{l c c c c c c}
\toprule
\multirow{2}{*}{Method}
& \multicolumn{3}{c}{GPT-4.1}
& \multicolumn{3}{c}{GPT-5} \\
\cmidrule(lr){2-4}\cmidrule(lr){5-7}
& Steps & Cost (\$) & Eff.
& Steps & Cost (\$) & Eff. \\
\midrule
OrcaLoca
& 20.22 & 0.46 & 1.48
& 36.93 & 0.75 & 1.16 \\
CoSIL
& 19.77 & 0.24 & 3.10 
& 19.52 & 0.31 & 2.64 \\
LocAgent
& 11.94 & 0.86 & 0.76
&  6.48 & 0.49 & 1.64 \\
\ours{}
& \textbf{6.75} & \textbf{0.18} & \textbf{4.63} 
& \textbf{6.34} & \textbf{0.22} & \textbf{4.15} \\
\bottomrule
\end{tabular}%
}
\vspace{-10pt}
\end{wraptable}


% % ================= Panel B =================
% \begin{table}[htbp]
% \centering
% \caption{Overall efficiency analysis (\textbf{Panel B}): Repository Reconstruction (Representational Efficiency).}
% \label{tab:efficiency_panel_b}
% \footnotesize
% \setlength{\tabcolsep}{4pt}

% \resizebox{\columnwidth}{!}{%
% \begin{tabular}{l c}
% \toprule
% \textbf{Representation} & \textbf{Input Tokens / Reduction} \\
% \midrule
% API Doc & 482,306 \quad / \quad -- \\
% \textbf{RPG (Ours)} & \textbf{55,060} \quad / \quad \textbf{-88.6\%} \\
% \bottomrule
% \end{tabular}%
% }
% \end{table}
\paragraph{RPG Facilitates Reasoning Efficiency.}
Table~\ref{tab:efficiency_analysis} evaluates the efficiency of agents guided by different substrates. Across all backbones, \ours{} achieves fewer steps and lower expenditure, yielding the highest cost-effectiveness (Acc@5/Cost). On GPT-5, \ours{} reaches an efficiency of 4.15 at a cost of \$0.22, whereas baselines such as OrcaLoca and LocAgent require higher expenditures for lower efficiency gains. This trend is consistent with GPT-4.1 results, where \ours{} attains the peak efficiency of 4.63. These results indicate that RPG-guided navigation enables precise exploration, concentrating reasoning resources on relevant code regions and reducing redundant API calls throughout the localization process.
% \vspace*{-5pt}
\subsection{Structural Evolvability}

\begin{figure}[htbp]
\vspace{0pt}
\centering

\begin{minipage}[t]{0.49\linewidth}
\vspace{0pt} 
\centering
\includegraphics[width=\linewidth]{figs/commit_cost.pdf}
\caption{Cost Efficiency Comparison: RPG Rebuilding versus Incremental Updates across Commit History.}
\label{fig:maintenance_cost}
\end{minipage}
\hfill
\begin{minipage}[t]{0.49\linewidth}
\vspace{15pt} 
\centering
\captionof{table}{Full vs. Incremental RPG Fidelity on SWE-bench Live. SWE-bench Live accuracy of RPGs across commits under full reconstruction (Full) and incremental maintenance (Incr.).}
\label{tab:full_incr_fidelity}
\footnotesize
\renewcommand{\arraystretch}{1.02}
\setlength{\tabcolsep}{3.0pt}

\resizebox{\linewidth}{!}{%
\begin{tabular}{@{}l l cccc cccc@{}}
\toprule
\multirow{2}{*}{Model} & \multirow{2}{*}{Strategy} &
\multicolumn{4}{c}{File-level} &
\multicolumn{4}{c}{Function-level} \\
\cmidrule(lr){3-6}\cmidrule(lr){7-10}
& & Acc@1 & Acc@5 & Pre & Rec & Acc@1 & Acc@5 & Pre & Rec \\
\midrule
\multirow{2}{*}{GPT-4o}
& Full
& \textbf{69.9} & \textbf{84.6} & \textbf{73.2} & 60.1
& \textbf{53.8} & 68.5 & \textbf{60.6} & 41.1 \\
& Incr.
& 69.2 & 83.5 & \textbf{73.2} & \textbf{60.3}
& 50.5 & \textbf{69.4} & 59.4 & \textbf{41.8} \\
\midrule
\multirow{2}{*}{GPT-4.1}
& Full
& \textbf{79.9} & 88.2 & \textbf{82.5} & \textbf{69.8}
& \textbf{67.4} & 80.3 & \textbf{73.3} & \textbf{55.4}  \\
& Incr.
& 78.0 & \textbf{90.5} & 81.4 & 69.0
& 64.7 & \textbf{81.9} & 72.1 & 52.6 \\
\bottomrule
\end{tabular}%
}
\end{minipage}


\vspace{-10pt}
\end{figure}


\paragraph{Incremental Maintenance Ensures Sustainable Scalability.}
To assess feasibility, we measure maintenance costs across a commit sequence. Figure~\ref{fig:maintenance_cost} shows that full reconstruction scales linearly and exceeds 14.7M tokens, whereas our incremental strategy uses only 633K tokens by isolating semantic deltas. This 95.7\% reduction confines heavy computation to a one-time initialization and effectively decouples ongoing maintenance costs from repository scale, enabling sustainable long-term operation.


\begin{figure*}[t]
    \centering
    \includegraphics[width=\linewidth]{figs/error_distribution.pdf}
    \caption{Distribution of Failure Modes on SWE-bench Verified. We analyze 100 failed trajectories per method with GPT-4o. Errors fall into four macro-groups: Tool \& Execution, Search \& Exploration, Reasoning \& Interpretation, and Context \& Scope, with 12 sub-types (T1–T12). See Appendix~\ref{app:err_analysis}.}
    \label{fig:error_dist}
    % \vspace*{-5pt}
\end{figure*}


% \input{tables/evolv}
\paragraph{Evolution Balance between Fidelity and Efficiency.} To validate resilience against semantic drift during updates, we assessed representational fidelity by deploying agents on SWE-bench Live using RPGs from both strategies. Table~\ref{tab:full_incr_fidelity} indicates that the "Incr." strategy maintains statistical parity with the "Full" baseline. Specifically, while "Incr." achieves slightly higher retrieval accuracy (81.9\% Acc@5 compared to 80.3\% for GPT-4.1), "Full" reconstruction retains a marginal edge, surpassing "Incr." by approximately 2\% in Precision and Recall. This balance confirms that our sustainable evolution effectively preserves the repository's semantic integrity with negligible degradation.

% \vspace*{-2pt}
\subsection{Agentic Navigability}

\begin{wraptable}[18]{r}{0.52\columnwidth}
    \centering
    \includegraphics[width=1.0\linewidth]{figs/agent_step.pdf}
    \caption{Impact of RPG Tooling on Agent Behavior on SWE-bench Verified. Step-wise action distributions induced by the RPG interface across LLMs.}
    \label{fig:agent_step}
    \vspace*{-10pt}
\end{wraptable}

\paragraph{RPG Induces Structured Exploration.}
To investigate whether RPG structures reasoning, we visualized tool usage distributions across LLMs. Figure~\ref{fig:agent_step} reveals a universal "Search-then-Zoom" pattern: agents prioritize broad topology traversal (ExploreRPG, SearchNode) to establish a global map before narrowing to fine-grained analysis (FetchNode). This trend is more pronounced in stronger reasoners (e.g., Claude-4.5), which leverage RPG's structural context to support extended interaction horizons. These results confirm that RPG effectively guides agents from global comprehension to localized implementation.

\paragraph{Dual-View Search Mitigates Navigational Failures.}
We manually analyzed 100 failed trajectories from GPT-4o to identify error patterns mitigated by the RPG structure. As shown in Figure~\ref{fig:error_dist}, RPG reduces Search \& Exploration failures compared to baselines. While systems like LocAgent and CoSIL utilize graph structures, they often suffer from Insufficient Coverage. \ours{} addresses this by providing dual-path access, where semantic features enable broad global retrieval to expand the search space, while the structured hierarchy guides the agent to reduce Redundant Search. This multi-view navigation ensures agents can accurately localize intent before traversing implementation-level dependencies. Improved localization also reduces downstream errors in Context \& Scope, keeping reasoning grounded in the correct implementation units.

