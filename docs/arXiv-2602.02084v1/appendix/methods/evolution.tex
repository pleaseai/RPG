% =========================
% A.2 Incremental Evolution
% =========================
\subsection{Incremental Evolution: Differential Update and Maintenance}
\label{app:evolution}

This subsection details how RPG-Encoder maintains the Repository Planning Graph (RPG) under continuous codebase
evolution. Given a repository update (e.g., a commit), our goal is to \emph{incrementally} update the semantic
hierarchy and its grounded mapping, ensuring the RPG remains a faithful semantic reflection of the codebase while
avoiding expensive full reconstruction. We formulate repository evolution as a stream of \textbf{atomic operations}:
\textsc{Delete}, \textsc{Modify}, and \textsc{Insert}. Each operation updates both (i) the local semantic
representation of affected entities and (ii) their placement within the feature hierarchy.

% -------------------------
% A.2.1 Differential Event Detection
% -------------------------
\subsubsection{Differential Event Detection and Operation Scheduling}
\label{app:evolution_diff}

\paragraph{From code diffs to semantic events.}
Given a code change $\Delta$ (e.g., a git diff between two revisions), we extract changed code entities at the
function/method granularity whenever possible. Each affected entity $u$ is categorized into one of three evolution events:
\begin{itemize}
    \item \textbf{Deletion:} $u$ is removed from the repository.
    \item \textbf{Modification:} $u$ exists in both revisions but its implementation changes.
    \item \textbf{Insertion:} $u$ is newly introduced in the new revision.
\end{itemize}
For \textsc{Modification}, we further distinguish between semantically stable edits and substantial semantic drift (Section~\ref{app:evolution_modify}), which determines whether the update can be handled locally or requires
structural relocation.

\paragraph{Scheduling principle.}
We schedule evolution operations under constraints that preserve structural consistency of the hierarchy,
prevent intermediate abstract nodes from accumulating dead branches,
and ensure that newly introduced entities do not disrupt the existing topological organization.

% -------------------------
% A.2.2 Deletion
% -------------------------
\subsubsection{Node Deletion with Structural Hygiene}
\label{app:evolution_delete}

\begin{algorithm}[h]
\caption{Incremental Deletion (Recursive Pruning)}
\label{alg:deletion}
\begin{algorithmic}[1]
\Require Current graph $G$, target node id $id$
\Ensure Updated graph $G$

\Function{DeleteNode}{$G, id$}
    \State $v \gets \Call{GetNode}{G, id}$
    \If{$v = \bot$}
        \State \Return $G$
    \EndIf

    \State $parent \gets v.\text{parent}$
    \State \Call{RemoveNode}{G, v} \Comment{Remove node and incident edges}
    \State \Call{PruneOrphans}{G, parent} \Comment{Structural hygiene}
    \State \Return $G$
\EndFunction

\Function{PruneOrphans}{$G, v$}
    \If{$v = \bot \ \lor\  \Call{IsRoot}{v}$}
        \State \Return
    \EndIf

    \If{$\Call{IsEmpty}{v.\text{children}}$}
        \State $gp \gets v.\text{parent}$
        \State \Call{RemoveNode}{G, v} \Comment{Prune empty abstract category}
        \State \Call{PruneOrphans}{G, gp} \Comment{Recurse upwards}
    \EndIf
\EndFunction
\end{algorithmic}
\end{algorithm}


\paragraph{Motivation.}
Deletion must maintain structural integrity of the hierarchy. Removing a leaf entity may render its ancestor abstract
nodes semantically vacuous (i.e., nodes that no longer cover any concrete code entities). Without cleanup, these dead
branches accumulate and reduce the signal-to-noise ratio for search and routing.

\paragraph{Recursive pruning.}
We enforce \textbf{structural hygiene} via bottom-up pruning: after removing a leaf node, we recursively delete any
ancestor abstract node whose subtree becomes empty. Pruning terminates once an ancestor still has remaining children
or once the root is reached.
This mechanism prevents stale semantic categories from persisting after refactors and ensures that the hierarchy remains compact and representative of the current repository state.

\paragraph{Algorithm.}
Algorithm~\ref{alg:deletion} specifies the deletion procedure and the recursive pruning logic.

% -------------------------
% A.2.3 Modification
% -------------------------
\subsubsection{Differential Modification Processing}
\label{app:evolution_modify}

\begin{algorithm}[htbp]
\caption{Differential Modification Handling}
\label{alg:modification}
\begin{algorithmic}[1]
\Require Graph $G$, file $f$, diff $\Delta$
\Ensure Updated graph $G$

\Function{ProcessModification}{$G, f, \Delta$}
    \State $\langle \mathcal{U}^+, \mathcal{U}^-, \mathcal{U}^{\sim} \rangle \gets \Call{ParseUnitDiff}{\Delta}$

    \Statex \textbf{1) Delete / Insert}
    \ForAll{$u \in \mathcal{U}^-$}
        \State $G \gets \Call{DeleteNode}{G, u.\text{id}}$
    \EndFor
    \ForAll{$u \in \mathcal{U}^+$}
      \State $G \gets \Call{InsertNode}{G, u, \textsc{LLMExtract}(u)}$
    \EndFor

    \Statex \textbf{2) Update / Re-route}
    \ForAll{$u \in \mathcal{U}^{\sim}$}
        \State $v \gets \Call{GetNode}{G, u.\text{id}};\; v.f \gets \Call{LLMUpdate}{u}$
        \If{$\Call{SemanticShift}{v} > \tau_{\text{drift}}$}
            \State $G \gets \Call{DeleteNode}{G, u.\text{id}}$ \Comment{logic drift}
           \State $G \gets \Call{InsertNode}{G, u, \textsc{LLMExtract}(u)}$
        \EndIf
    \EndFor

    \State \Return $G$
\EndFunction
\end{algorithmic}
\end{algorithm}

\paragraph{Motivation.}
A code edit may either preserve the original intent (e.g., bug fixes, refactoring, parameter tuning) or substantially
change functionality (semantic drift). Treating both cases identically is suboptimal: in-place updates are sufficient
for minor edits, while major drift requires relocating the entity to a semantically congruent domain.

\paragraph{Minor update vs.\ semantic drift.}
Given a modified entity $u$ with old/new versions $(u^{old}, u^{new})$, we compute semantic features
$f(u^{old})$ and $f(u^{new})$ using the same parsing constraints as in extraction. We then assess drift based on:
(i) feature overlap/consistency, and (ii) an LLM judgement constrained by explicit criteria. If drift is minor, we perform an in-place update
of the node’s semantic summary; otherwise, we trigger re-routing.

\paragraph{Re-routing as composition.}
When semantic drift is significant, we treat modification as a composition of atomic operations:
\[
\textsc{Modify}(u) \;\Rightarrow\; \textsc{Delete}(u^{old}) + \textsc{Insert}(u^{new}),
\]
which relocates the entity to a new functional domain via the same semantic routing procedure used for insertion.
This guarantees that the hierarchy reflects the updated intent rather than only updating text summaries in an
incorrect domain.

\paragraph{Algorithm.}
Algorithm~\ref{alg:modification} formalizes the differential modification procedure, including the branching logic
between in-place update and re-routing.

% -------------------------
% A.2.4 Insertion
% -------------------------
\subsubsection{Node Insertion via Semantic Routing}
\label{app:evolution_insert}


\begin{algorithm}[h]
\caption{Incremental Additions (LLM-Based Semantic Routing)}
\label{alg:insertion}
\begin{algorithmic}[1]
\Require Current Graph $G$, New Unit $u$, Feature Summary $f_u$
\Ensure Updated Graph $G$ with $u$ inserted

\Function{InsertNode}{$G, u, f_u$}
    \State $v_{\text{best}} \gets \Call{FindBestParent}{G.\text{root}, f_u}$
    \State $v_{\text{new}} \gets \Call{CreateNode}{u, f_u}$
    \State \Call{AddEdge}{$G, v_{\text{best}}, v_{\text{new}}, \mathcal{E}_{\text{feature}}$}
        \Comment{Attach to semantically determined parent}
    \State \Return $G$
\EndFunction

\Statex
\Function{FindBestParent}{$v_{\text{curr}}, f_{\text{target}}$}
    \State $Candidates \gets \{\,c \in \Call{Children}{v_{\text{curr}}} \mid \Call{IsAbstract}{c}\,\}$
    
    \If{$Candidates = \emptyset$}
        \State \Return $v_{\text{curr}}$ \Comment{Base case: No deeper abstract categories}
    \EndIf

    \State \Comment{Prompt LLM to select the best functional fit among children}
    \State $Context \gets \{ (c.\text{id}, c.f) \mid c \in Candidates \}$
    \State $v_{\text{choice}} \gets \Call{LLM\_Route}{Context, f_{\text{target}}}$
    
    \If{$v_{\text{choice}} \neq \text{null}$}
        \State \Return $\Call{FindBestParent}{v_{\text{choice}}, f_{\text{target}}}$
            \Comment{LLM chose a branch, drill down recursively}
    \Else
        \State \Return $v_{\text{curr}}$ \Comment{LLM decided no child is a better fit}
    \EndIf
\EndFunction
\end{algorithmic}
\end{algorithm}


\paragraph{Motivation.}
Naively attaching new entities to the root (or a fixed default module) breaks the semantic organization of the RPG and
degrades downstream navigation. Instead, we treat insertion as a \textbf{semantic placement} problem: find the most
appropriate abstract parent node in the current feature hierarchy that best matches the new entity’s functionality.

\paragraph{Routing objective.}
Let $u$ be a newly added code entity with semantic features $f(u)$. Starting from the root of the feature hierarchy,
we iteratively select the child domain whose description best aligns with $f(u)$, drilling down until no more
meaningful specialization is possible. This \textbf{top-down semantic routing} ensures that $u$ is inserted into the
most specific functional domain available while preserving interpretability of the hierarchy.

\paragraph{Algorithm.}
Algorithm~\ref{alg:insertion} formalizes the insertion procedure. At each step, the router considers the candidate
children of the current node and chooses the best target; if no child is sufficiently compatible, the algorithm
terminates and inserts $u$ at the current level. This prevents over-forcing entities into unrelated subtrees.

\paragraph{Complexity and scalability.}
Incremental evolution in RPG-Encoder is inherently local.
Each atomic operation affects only a bounded region of the hierarchy, without requiring global reconstruction.
As a result, maintenance cost scales with the \emph{magnitude of the change} rather than the size of the repository, enabling efficient and stable synchronization of the RPG under continuous development.
