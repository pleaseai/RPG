% =========================
% A.3 RPG Operation
% =========================
\subsection{RPG Operation: Agentic Tool-use and Navigation Logic}
\label{app:operation}

This subsection details how RPG is operationalized as an actionable substrate for repository understanding.
Beyond serving as a semantic representation, RPG exposes a \emph{tool interface} that bridges high-level intents
to concrete code entities and their dependency contexts. Concretely, we provide three complementary tools:
\textbf{SearchNode} for intent-based discovery, \textbf{FetchNode} for precision context retrieval, and
\textbf{ExploreRPG} for structural traversal on the RPG topology.

% -------------------------
% A.3.1 Tool Interfaces
% -------------------------
\subsubsection{Tool Interfaces and Prompt Specifications}
\label{app:operation_interfaces}

\paragraph{Design principles.}
The tool suite is designed to support a common agent workflow in repository understanding:
(i) start from vague or behavioral intents and obtain candidate code anchors;
(ii) verify anchors with precise source context; and
(iii) expand locally to cover call chains and related components.
To ensure tool outputs are deterministic and machine-consumable, each tool prompt defines a strict parameter schema
and return format.

\paragraph{SearchNode: intent-based discovery.}
\textbf{SearchNode} unifies \emph{semantic discovery} and \emph{textual retrieval}. It supports three modes:
\texttt{features} (intent $\rightarrow$ feature nodes / mapped code entities),
\texttt{snippets} (keyword/symbol search over the repository),
and \texttt{auto} (feature mapping first, followed by snippet search when needed).
Importantly, \texttt{search\_scopes} can restrict the search to selected feature subtrees, leveraging the grounded
hierarchy constructed in Appendix~\ref{app:extraction} to improve precision.

% Option A: move your existing tcblisting into separate files and input here
\begin{tcblisting}{
  title={SearchNode Tool Prompt},
  colback=lightgray,
  colframe=black,
  arc=1mm,
  boxrule=1pt,
  left=1mm,right=1mm,top=1mm,bottom=1mm,
  breakable,
  fontupper=\scriptsize\ttfamily,
  listing only,
  listing engine=listings,
  listing options={
    breaklines=true,
    breakatwhitespace=false,
    breakindent=0pt,
    prebreak=\mbox{},
    postbreak=\mbox{},
    keepspaces=true,
    columns=fullflexible,
    tabsize=4
  }
}
## Tool Name: SearchNode
### Description
Unified search tool for repository navigation. Use it to (1) map high-level functional/behavioral descriptions to concrete code entities via RPG mapping, and/or (2) retrieve concrete code snippets via symbol/file/keyword search. Prefer behavior-to-code mapping when you don't know the exact file/class/function name; then narrow down with snippet search.
Tip: Avoid vague terms; use concrete behavior phrases or high-signal identifiers.
### Parameters
{
  "tool_name": "SearchNode",
  "parameters": {
    "mode": "<'features' | 'snippets' | 'auto'. Required. 'auto' may run both: feature-mapping first, then snippet search.>",
    "feature_terms": "<List of concrete behavioral/functionality phrases. Required when mode is 'features' or 'auto'.>",
    "search_scopes": "<List of valid feature entity paths to restrict the Functionality SubGraph. Optional.>",
    "search_terms": "<List of file paths, qualified entities (file:Class.method), or high-signal text keywords. Required when mode is 'snippets' or when 'auto' proceeds to snippet search.>",
    "line_nums": "<Two integers [start, end] to extract lines from a specific file. Requires an exact file path. Optional.>",
    "file_path_or_pattern": "<File path or glob pattern to restrict snippet search. Default: '**/*.py'>",
  }
}
### Returns
- If feature search runs: matched feature nodes mapped to code entities (feature name, code entity, file path, line range when available)
- If snippet search runs: matched code snippets, complete files, or located entities based on search terms / line ranges
\end{tcblisting}

\paragraph{FetchNode: precision retrieval and verification.}
\textbf{FetchNode} retrieves exact source context and metadata for known candidates (code entities or feature paths).
It is used as a verification step after discovery to ensure the agent reasons on faithful code snippets rather than
speculative guesses. FetchNode returns file paths, line ranges, entity types, mapped feature information, and a code preview.

\begin{tcblisting}{
  title={FetchNode Tool Prompt},
  colback=lightgray,
  colframe=black,
  arc=1mm,
  boxrule=1pt,
  left=1mm,right=1mm,top=1mm,bottom=1mm,
  breakable,
  fontupper=\scriptsize\ttfamily,
  listing only,
  listing engine=listings,
  listing options={
    breaklines=true,
    breakatwhitespace=false,
    breakindent=0pt,
    prebreak=\mbox{},
    postbreak=\mbox{},
    keepspaces=true,
    columns=fullflexible,
    tabsize=4
  }
}
## Tool Name: FetchNode
### Description
- Retrieve precise metadata and source context for known code or feature entities.
- Use this tool to verify candidate code locations after identifying them through searches or graph exploration.
- Returns exact file path, entity type, start/end lines, mapped feature information, and a code preview.

### Parameters
{
  "tool_name": "FetchNode",
  "parameters": {
    "code_entities": "<List of existing and validated code entities in the current repository; non-existent paths or speculative entities may be ignored. Optional.>",
    "feature_entities": "<List of existing and validated feature paths in the current repository; non-existent entries may be ignored. Optional.>"
  }
}
### Returns
- Entity type (file/class/method/feature)
- Feature paths and code content (with source context / preview)
- Start/end lines and mapped feature information (when available)
\end{tcblisting}

\paragraph{ExploreRPG: topological traversal.}
\textbf{ExploreRPG} exposes the structural connectivity of RPG, enabling traversal along dependency edges
(\texttt{imports}, \texttt{invokes}, \texttt{inherits}, etc.) and/or containment/composition relations.
Starting from validated anchors, the agent can traverse upstream/downstream to uncover dependencies, impacted components,
and semantically related regions.

\begin{tcblisting}{
  title={ExploreRPG Tool Prompt},
  colback=lightgray,
  colframe=black,
  arc=1mm,
  boxrule=1pt,
  left=1mm,right=1mm,top=1mm,bottom=1mm,
  breakable,
  fontupper=\scriptsize\ttfamily,
  listing only,
  listing engine=listings,
  listing options={
    breaklines=true,
    breakatwhitespace=false,
    breakindent=0pt,
    prebreak=\mbox{},
    postbreak=\mbox{},
    keepspaces=true,
    columns=fullflexible,
    tabsize=4
  }
}
## Tool Name: ExploreRPG
### Description
- Explore call chains and functional paths in the Repository Planning Graph.
- Starting from known code or feature entities, traverse upstream/downstream to discover related functions, files, and feature nodes.
### Parameters
{
  "tool_name": "ExploreRPG",
  "parameters": {
    "start_code_entities": "<Optional list of existing code entities in the current repository (file paths, classes, functions, or qualified names). Non-existent/speculative entities may be ignored or rejected.>",
    "start_feature_entities": "<Optional list of existing feature paths in the current repository. Non-existent entries may be ignored or rejected.>",
    "direction": "<Traversal direction: 'upstream' (dependencies), 'downstream' (dependents), or 'both'. Default: 'downstream'.>",
    "traversal_depth": "<Maximum traversal depth. Default: 2. Use -1 for unlimited depth.>",
    "entity_type_filter": "<Optional filter restricting traversal node types. Valid values: 'directory', 'file', 'class', 'function', 'method'.>",
    "dependency_type_filter": "<Optional filter restricting dependency edge types. Valid values: 'composes', 'contains', 'inherits', 'invokes', 'imports'.>"
  }
}
### Returns
- Connected nodes and edges (code or feature view)
- Hints for invalid or fuzzy matches
\end{tcblisting}

% -------------------------
% A.3.2 Tool-use Policy
% -------------------------
\subsubsection{Tool-use Policy for Repository Understanding}
\label{app:operation_policy}

\paragraph{Canonical tool orchestration.}
We adopt a simple and robust orchestration policy that prioritizes semantic grounding before reading large contexts.
Given a natural-language intent $\mathcal{I}$, the agent executes:

\begin{enumerate}
    \item \textbf{Semantic discovery (SearchNode/features or auto):}
    convert $\mathcal{I}$ into concrete behavioral terms and retrieve candidate feature nodes and mapped code entities.
    If available, supply \texttt{search\_scopes} to restrict discovery to the most relevant functional subtrees.

    \item \textbf{Precision verification (FetchNode):}
    for top candidates, fetch exact code context (file path + line range + preview) and confirm semantic compatibility.
    Candidates that cannot be verified are discarded.

    \item \textbf{Local expansion (ExploreRPG):}
    from verified anchors, traverse dependency edges (e.g., \texttt{invokes}, \texttt{imports}) to recover call chains,
    utilities, and related modules. This step is used to (i) locate the root cause, (ii) map the impact surface, or
    (iii) identify integration points.

    \item \textbf{Pinpoint retrieval (optional SearchNode/snippets):}
    if the target remains ambiguous, run snippet search with high-signal identifiers obtained from previous steps
    (exact symbols, file paths, error strings), optionally extracting specific line ranges.
\end{enumerate}

\paragraph{Fallback rules.}
When semantic discovery returns insufficient recall (e.g., missing/weak feature matches), the agent falls back to
\texttt{snippets} mode to bootstrap concrete anchors, then returns to \textbf{FetchNode} and \textbf{ExploreRPG}.
When snippet search yields too many matches, the agent tightens constraints by adding (i) feature scopes,
(ii) file path patterns, or (iii) symbol-qualified queries.

This policy minimizes wasted context and reduces hallucination risk:
SearchNode provides intent-to-code grounding, FetchNode ensures the agent reasons on exact source, and ExploreRPG
reveals topological structure that cannot be reliably inferred from local snippets alone.

% -------------------------
% A.3.3 Execution Traces / Examples
% -------------------------
\subsubsection{Execution Traces and Examples}
\label{app:operation_examples}

We illustrate the practical efficacy of these tools through the execution traces shown in
Figure~\ref{fig:tool_execution_examples}. These traces demonstrate how the agent navigates from abstract intents to
specific code implementations, leveraging both the semantic hierarchy and the dependency topology of RPG.

\begin{figure}[htbp]
    \centering
    \includegraphics[width=\linewidth]{figs/tool_example.pdf}
    \caption{Execution traces of the three primary agentic tools. \textbf{SearchNode} maps abstract intent to concrete code;
    \textbf{FetchNode} retrieves precise source context; and \textbf{ExploreRPG} reveals topological connections and call relations.}
    \label{fig:tool_execution_examples}
\end{figure}

As depicted in the figure, each tool provides distinct structural signals that support the agent's reasoning:

\begin{itemize}
    \item \textbf{SearchNode (Left):}
    demonstrates intent-to-code grounding by mapping a behavioral query (e.g., ``expression serialization'') to a
    concrete code entity and its associated feature description. This step transforms ambiguous intent into executable anchors.

    \item \textbf{FetchNode (Center):}
    retrieves precise source context for a candidate entity (e.g., \texttt{\_check\_vector}), including exact line ranges
    and a preview snippet, enabling verification and preventing reasoning on speculative locations.

    \item \textbf{ExploreRPG (Right):}
    traverses the RPG topology from a verified anchor (e.g., \texttt{kernS}) to expose invocation and dependency relations.
    By showing edges such as \texttt{invokes} and their connected nodes, the agent can recover call chains and impacted modules,
    supporting systematic debugging and repository-level understanding.
\end{itemize}
