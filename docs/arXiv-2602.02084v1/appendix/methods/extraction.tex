% =========================
% A.1 RPG Extraction
% =========================
\subsection{RPG Extraction: Semantic Lifting and Hierarchical Encoding}
\label{app:extraction}

This subsection details the construction stage of RPG-Encoder, which transforms a raw repository into a
hierarchically organized \emph{feature space} (the semantic backbone of RPG) together with a \emph{grounded}
mapping that links abstract functional nodes to concrete directory scopes.
Concretely, the extraction stage proceeds in three steps:
(1) \textbf{Semantic lifting} that converts low-level code entities into atomic functional features;
(2) \textbf{Latent architecture recovery} that reorganizes these features into a consistent three-level hierarchy;
and (3) \textbf{Artifact grounding} that anchors each abstract subtree to a compact set of physical directory paths.
The resulting hierarchy serves as the \emph{Functionality SubGraph} used by downstream agentic tools (Appendix~\ref{app:operation}).

% -------------------------
% A.1.1 Semantic Lifting
% -------------------------
\subsubsection{Semantic Lifting via Prompted Semantic Parsing}
\label{app:extraction_semantic}

\paragraph{Global parsing strategy.}
Given a repository $\mathcal{R}$, semantic lifting is performed from a \emph{global perspective} rather than on
individual files in isolation. We first identify all code entities of interest, including classes, methods and functions, and treat them as the fundamental semantic units to be analyzed.
This global view allows the model to maintain consistent semantic granularity across the repository and reduces
local biases introduced by file boundaries.

\paragraph{Semantic units and batching.}
To accommodate repositories of varying scales while respecting model context limits, code entities are abstracted into \emph{semantic units} and analyzed in batches under a controlled token budget. Each semantic unit represents a coherent functional entity,
ensuring that semantically coupled components are interpreted in context.
Batches are constructed to balance completeness and efficiency, such that every semantic unit is analyzed exactly once while enabling scalable processing of large repositories.

\paragraph{Semantic feature representation.}
For each code entity $u$, the parser produces a set of \emph{atomic semantic features}
$f(u)=\{a_1, a_2, \dots\}$, where each $a_i$ is a short verb--object phrase describing \emph{what} the entity does rather than \emph{how} it is implemented. These atomic features are intentionally constrained to be:
(i) \textbf{single-responsibility}, (ii) \textbf{implementation-agnostic}, and (iii) \textbf{lexically normalized}
(lowercase English, concise phrasing). This normalization is critical for subsequent routing and hierarchical encoding,
since it provides stable semantic anchors for grouping and comparison across the repository.

\paragraph{Prompt template (semantic parsing).}
We implement semantic lifting using the following prompt template, which enforces: (1) complete coverage of all
functions in the chunk, (2) strict output schema, and (3) feature naming rules that avoid vague verbs and
implementation details. The prompt returns a JSON object mapping each function name to a list of semantic features.

\begin{tcblisting}{
  title={Semantic Parsing Prompt},
  colback=lightgray,
  colframe=black,
  arc=1mm,
  boxrule=1pt,
  left=1mm,right=1mm,top=1mm,bottom=1mm,
  breakable,
  fontupper=\scriptsize\ttfamily,
  listing only,
  listing engine=listings,
  listing options={
    breaklines=true,
    breakatwhitespace=false,
    breakindent=0pt,
    prebreak=\mbox{},
    postbreak=\mbox{},
    keepspaces=true,
    columns=fullflexible,
    tabsize=4
  }
}
## Instruction
You are a senior software analyst.
Your goal is to analyze all functions in the current input and return their key semantic features -- what each function does, not how it's implemented.

### Key Goals
- Complete analysis: Provide semantic feature extraction for every function in the given input. Do not skip any function.
- Batch perspective: Analyze all functions in the chunk together, considering their roles within the overall system.
- High-level behavior: Focus on the purpose and role of each function, not on low-level implementation details.
- If multiple definitions share the same method name (e.g., property getter and setter for the same attribute), you may output that method name only once and merge their semantic features; you do not need to distinguish decorator variants.

## Feature Extraction Principles
Follow these principles when analyzing functions:
1. Focus on the purpose and behavior of the function -- what role it serves in the system.
2. Do NOT describe implementation details, variable names, or internal logic such as loops, conditionals, or data structures.
3. If a function performs multiple responsibilities, break them down into separate features.
4. Use your understanding of each function's name, signature, and code to infer its intent.
5. Only analyze functions included in the current input -- do not guess or invent other functions.
6. Do not omit any function, including utility or helper functions.

### Feature Naming Rules:
1. Use verb + object format (e.g., `load config`, `validate token`).
2. Use lowercase English only.
3. Describe purpose not implementation (focus on what, not how).
4. Each feature must express one single responsibility.
5. If a method has multiple responsibilities, split into multiple atomic features.
6. Keep features short and atomic (prefer 3--8 words; no full sentences; no punctuation).
7. Avoid vague verbs (`handle`, `process`, `deal with`); prefer precise verbs (`load`, `validate`, `convert`, `update`, `serialize`, `compute`, `check`, `transform`).
8. Avoid implementation details (no loops, conditionals, data structures, control flow).
9. Avoid libraries/frameworks/formats (say `serialize data`, not `pickle object` / `save to json`).
10. Prefer domain/system semantics over low-level actions (`manage session` > `update dict`).
11. Avoid chaining actions (don't write `initialize config and register globally`; split into separate features).

## Output Format
You must respond with the following structure:
A `<solution>` block -- a JSON object mapping each function name to a list of its semantic features.
If a function does not implement any meaningful features (e.g., it's a stub), still include it with an empty list.

### Output Template:
<solution>
{{
  "func_name_1": ["feature one", "feature two"],
  "func_name_2": [],
  ...
}}
</solution>

## Input Context
### Repository Name
<repo_name>
{repo_name}
</repo_name>

### Repository Overview
<repo_info>
{repo_info}
</repo_info>
\end{tcblisting}

\paragraph{Post-processing and validation.}
We apply lightweight validation to guarantee the output is machine-consumable:
(i) JSON parsing and schema checking (every function in the input must appear as a key);
(ii) feature list normalization (whitespace, casing, deduplication);
and (iii) optional merging for decorator variants (e.g., property getter/setter) when they share the same method name,
as allowed by the prompt. If the model returns malformed output, we retry with a minimal format correction instruction
without changing semantic constraints.

\paragraph{Illustrative example.}
Figure~\ref{fig:parsing_process} shows an end-to-end example of semantic lifting, where raw code snippets are mapped
to their corresponding atomic semantic features.

\begin{figure}[htbp]
    \centering
    \includegraphics[width=\linewidth]{figs/parsing_example.pdf}
    \caption{Illustration of raw code snippets and their corresponding semantic features extracted via semantic parsing.}
    \label{fig:parsing_process}
\end{figure}

% -------------------------
% A.1.2 Latent Architecture Recovery
% -------------------------
\subsubsection{Latent Architecture Recovery for Hierarchical Encoding}
\label{app:extraction_latent}

\paragraph{Motivation.}
Semantic lifting yields a set of fine-grained features distributed across many files, which is insufficient as a
planning substrate: flat features are hard to navigate, while directory-only grouping often overlooks logical roles.
We therefore recover a \emph{latent functional architecture} that reorganizes the repository into a consistent,
interpretable, and searchable hierarchy. We enforce a strict \textbf{three-level} feature path format:
\[
\texttt{<functional area>/<category>/<subcategory>},
\]
which balances abstraction (top-level intent) and specificity (fine-grained specialization), while keeping routing and
tool-based navigation tractable.

\paragraph{Step 1: Domain discovery (functional areas).}
We first discover a small set of high-level functional areas that act as architectural centroids. The model is guided
to propose meaningful areas (e.g., \texttt{DataProcessing}, \texttt{ModelTraining}, \texttt{EvaluationMetrics}) while
avoiding low-signal directories such as vendor code, tests, or documentation.

\begin{tcblisting}{
  title={Domain Discovery Prompt},
  colback=lightgray,
  colframe=black,
  arc=1mm,
  boxrule=1pt,
  left=1mm,right=1mm,top=1mm,bottom=1mm,
  breakable,
  fontupper=\scriptsize\ttfamily,
  listing only,
  listing engine=listings,
  listing options={
    breaklines=true,
    breakatwhitespace=false,
    breakindent=0pt,
    prebreak=\mbox{},
    postbreak=\mbox{},
    keepspaces=true,
    columns=fullflexible,
    tabsize=4
  }
}
## Instructions
You are an expert software architect and repository analyst.
Your goal is to analyze the repository holistically and identify its main functional areas -- coherent, high-level modules or subsystems that reflect the repository's architecture and purpose.

### Guidelines
- Think from a software architecture perspective; group code into major, distinct responsibilities (e.g., data loading/processing, training/inference, evaluation/metrics, visualization/reporting, APIs/interfaces, configuration/utilities/infrastructure).
- Avoid listing individual files or small helpers, third-party/vendor code, and build/test/docs directories.
- Ensure each area is meaningful and represents a clear responsibility in the codebase.

### Naming Principles
- Single Responsibility: Each area should cover one logical concern (e.g., "DataProcessing", "ModelTraining").
- High-Level Abstraction: Group related submodules; separate distinct layers.
- Consistency: Use PascalCase for names (e.g., "FeatureExtraction", "EvaluationMetrics").
- Meaningful Scope:
  - Merge closely related components (e.g., "data_loader", "dataset" -> "DataProcessing")
  - Avoid vague terms like "core", "misc", "other"
  - Use domain-specific names when appropriate (e.g., "TextPreprocessing", "ImageSegmentation")

### Output Format
Return only the result in this exact format:
<solution>
[
"functional_area1", "functional_area2", "functional_area3", ...
]
</solution>
\end{tcblisting}

\paragraph{Step 2: Hierarchical construction (three-level paths).}
Given the discovered functional areas and the parsed feature groups, we perform hierarchical construction by assigning
each top-level feature group to a unique three-level target path. This step is formulated as a constrained semantic
assignment problem: the model must use only the provided functional areas for the first level, and it must generate
intent-focused category/subcategory labels following the same semantic naming rules used in semantic lifting.

\begin{tcblisting}{
  title={Hierarchical Construction Prompt},
  colback=lightgray,
  colframe=black,
  arc=1mm,
  boxrule=1pt,
  left=1mm,right=1mm,top=1mm,bottom=1mm,
  breakable,
  fontupper=\scriptsize\ttfamily,
  listing only,
  listing engine=listings,
  listing options={
    breaklines=true,
    breakatwhitespace=false,
    breakindent=0pt,
    prebreak=\mbox{},
    postbreak=\mbox{},
    keepspaces=true,
    columns=fullflexible,
    tabsize=4
  }
}
## Instruction
You are an expert software architect and large-scale repository refactoring specialist.

## Goal
Reorganize and enrich the repository's parsed feature tree by assigning each top-level feature group
(e.g., "data_loader", "model_trainer", "metrics") to the most semantically appropriate location
within the target architecture.

## Target Path Format (STRICT)
Each target path must have exactly three levels:
`<functional_area>/<category_level_1>/<subcategory_level_2>`
- `functional_area` must be one of the provided <functional_areas>.
- `category_level_1` expresses broader purpose or lifecycle role.
- `subcategory_level_2` adds precise specialization or context.
- Each segment: concise (2--5 words), semantically meaningful, intent-focused.
Examples:
- "data ingestion/pipeline orchestration/task scheduling"
- "model training/optimization strategy/hyperparameter tuning"
Avoid filler labels (e.g., "misc", "others", "core", "general").

## Semantic Naming Rules
When creating or adjusting semantic labels (categories/subcategories), follow:
1. Use "verb + object" phrasing; e.g., `load config`, `validate token`.
2. Use lowercase English only.
3. Describe purpose, not implementation.
4. Ensure each label expresses a single responsibility.
5. When multiple distinct roles exist, use multiple precise labels rather than one overloaded label.
6. Avoid vague verbs such as `handle`, `process`, and `deal with`.
7. Avoid implementation details, including control-flow or data-structure references.
8. Avoid mentioning specific libraries, frameworks, or formats; prefer `serialize data` over `pickle object` or `save to json`.
9. Prefer domain or system semantics over low-level actions; use `manage session` rather than `update dict`.

## Scope Constraints
- Only assign top-level groups (keys of <parsed_folder_tree>).
- Exclude docs/examples/tests/vendor code unless essential to core functionality.
- Do not invent new functional areas; use only those in <functional_areas>.
- You may define new categories/subcategories as needed, but they must remain meaningful and consistent.

## Output Format (STRICT)
Return only the JSON object wrapped exactly as:
<solution>
{
  "<functional_area>/<category>/<subcategory>": ["top_level_group_1", "top_level_group_2", ...],
  "<functional_area>/<category>/<subcategory>": ["top_level_group_3", ...]
}
</solution>
\end{tcblisting}

\paragraph{Outputs and usage.}
The output of hierarchical construction is a mapping from feature paths to sets of feature groups, which induces a
topological feature tree $\mathcal{T}_{\text{feature}}$. This tree serves two purposes:
(i) it provides high-signal \emph{search scopes} for intent-to-code mapping, and
(ii) it supports routing and traversal by ensuring semantically coherent boundaries at each level.

\paragraph{Illustrative examples.}
Figures~\ref{fig:domain_discovery_example} and~\ref{fig:hierarchical_construction} provide examples of domain discovery
and hierarchical construction, respectively.

\begin{figure}[htbp]
    \centering
    \includegraphics[width=\linewidth]{figs/domain_example.pdf}
    \caption{Illustrative example of the Domain Discovery phase.}
    \label{fig:domain_discovery_example}
\end{figure}

\begin{figure}[htbp]
    \centering
    \includegraphics[width=\linewidth]{figs/recovery_example.pdf}
    \caption{Example of the Hierarchical Construction phase.}
    \label{fig:hierarchical_construction}
\end{figure}

% -------------------------
% A.1.3 Artifact Grounding
% -------------------------
\subsubsection{Artifact Grounding: Anchoring Abstract Subtrees to Directory Scopes}
\label{app:extraction_grounding}

\paragraph{Problem formulation.}
To bridge the semantic hierarchy $\mathcal{T}_{\text{feature}}$ with physical repository artifacts, we ground each
abstract node $v$ to a compact set of directory scopes.
Let $\mathcal{L}(v)$ denote the set of leaf nodes in the subtree rooted at $v$.
For each leaf node $l \in \mathcal{L}(v)$, let $\text{path}(l)$ be its physical file path.
We define the \textbf{File Coverage} $\mathcal{C}(v)$ as the collection of parent directories for all leaves under $v$:
\begin{equation}
\mathcal{C}(v) = \{ \text{dir}(\text{path}(l)) \mid l \in \mathcal{L}(v) \},
\end{equation}
where $\text{dir}(\cdot)$ extracts the directory component of a file path.
We seek a compact representation $\hat{\pi}(v)$ that succinctly covers $\mathcal{C}(v)$ while preserving functional
boundaries across distinct modules.

\paragraph{Bottom-up propagation with Trie-based branching analysis.}
A naive common-prefix (LCA) computation may over-collapse unrelated modules into overly general roots (e.g., \texttt{/}).
To avoid this, we compute $\hat{\pi}(v)$ via a bottom-up propagation strategy that aggregates coverage and then
simplifies it using a Trie-based branching analysis: all paths in $\mathcal{C}(v)$ are inserted into a Prefix Tree,
and only \textbf{branching nodes} (multiple children or path termination) are retained as grounded scopes.
This yields a minimal set of directory LCAs that dominate disjoint coverage regions while respecting module boundaries.
Algorithm~\ref{alg:metadata-propagation} provides the full procedure.


\begin{algorithm}[h]
\caption{Bottom-Up Path Metadata Propagation}
\label{alg:metadata-propagation}
\begin{algorithmic}[1]
\Require Feature Tree $T = (V, E)$, Leaf paths $\text{path}(\cdot)$
\Ensure Grounded path assignments $\hat{\pi}(v)$ for all $v \in V$

\State \textbf{Function} \textsc{Propagate}$(v)$:
    \If{$v$ is Leaf}
        \State \Return $\{ \text{dir}(\text{path}(v)) \}$ \Comment{Base case: Return physical directory}
    \EndIf
    
    \State $\mathcal{S} \leftarrow \emptyset$
    \For{$child \in \text{Children}(v)$}
        \State $\mathcal{S} \leftarrow \mathcal{S} \cup \textsc{Propagate}(child)$ \Comment{Recursively aggregate child coverage}
    \EndFor
    
    \State $\hat{\pi}(v) \leftarrow \textsc{ComputeLCA}(\mathcal{S})$ \Comment{Abstract concrete paths into logical scopes}
    \State \Return $\mathcal{S}$ \Comment{Propagate full coverage to upper layers}
\State \textbf{End Function}

\Statex
\State \textbf{Function} \textsc{ComputeLCA}$(\mathcal{S})$:
    \State $Trie \leftarrow \textsc{BuildTrie}(\mathcal{S})$ \Comment{Construct Prefix Tree from path set}
    \State $L \leftarrow \emptyset$
    \For{$node \in \textsc{PostOrder}(Trie)$} \Comment{Bottom-up traversal for optimal pruning}
        \If{$node.\text{is\_branching}()$ \textbf{or} $node.\text{is\_terminal}()$}
             \State $L.\text{add}(node.\text{path})$ \Comment{Identify meaningful functional boundary}
             \State \textsc{PruneSubtree}$(node)$ \Comment{Consolidate redundant sub-paths}
        \EndIf
    \EndFor
    \State \Return $L$
\State \textbf{End Function}
\end{algorithmic}
\end{algorithm}

\paragraph{Complexity analysis.}
Let $N$ be the number of paths and $L$ the maximum directory depth.
Trie construction and branching-node extraction take $O(N\cdot L)$ time, bounded by total path characters.
Since propagation visits each feature node once in a bottom-up pass, the total grounding overhead scales linearly with
repository size and is negligible compared with LLM inference.